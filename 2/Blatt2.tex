\documentclass[a4paper,10pt,oneside,leqno]{scrartcl}
%\usepackage[paper=a4paper,includefoot,includehead,left=30mm,right=20mm,top=20mm,bottom=20mm]{geometry}
\usepackage[utf8]{inputenc}
\usepackage{listingsutf8}
\usepackage[pdftex]{graphicx}
%\usepackage[ngerman]{babel}
\usepackage{url}
\usepackage{hyperref}
\usepackage{amsfonts}
\usepackage{amssymb}
\usepackage{amsmath}
\usepackage{tikz}
\usepackage{listings}
\usepackage{textcomp}
\usepackage{amsmath}
%\usepackage{lipsum}%%%%%%%%%%%%%%%%%%%%%%%%%%%%%%%%%%%%%%%%%%%%%%%%%%%%%%%%%%%%%%%%%%%%%%%%%%%%%%%%%%%%
\usepackage{mathtools}
\usepackage{amsfonts}
\usepackage{amssymb}
\usepackage{amsmath}
\usepackage{booktabs}
%usepackage[utf8x]{inputenc}
\usepackage[T1]{fontenc}
\usepackage{lmodern} %Latin modern = enhanced CM font
\usepackage{xspace} %Space enhancements
%\usepackage{algorithmic}%%%%%%%%%%%%%%%%%%%%%%%%%%%%%%%%%%%%%%%%%%%%%%%%%%%%%%%%%%%%%%%%%%%%%%%%%%%%%%%%%%
\usepackage{algpseudocode}
\renewcommand{\thefootnote}{\fnsymbol{footnote}}
\definecolor{mygreen}{rgb}{0,0.6,0}
\definecolor{mygray}{rgb}{0.5,0.5,0.5}
\definecolor{mymauve}{rgb}{0.58,0,0.82}
\lstset{ %
  backgroundcolor=\color{white},   % choose the background color; you must add \usepackage{color} or \usepackage{xcolor}
  basicstyle=\footnotesize,        % the size of the fonts that are used for the code
  breakatwhitespace=false,         % sets if automatic breaks should only happen at whitespace
  breaklines=true,                 % sets automatic line breaking
  captionpos=b,                    % sets the caption-position to bottom
  commentstyle=\color{mygreen},    % comment style
  deletekeywords={...},            % if you want to delete keywords from the given language
  escapeinside={\%*}{*)},          % if you want to add LaTeX within your code
  extendedchars=true,              % lets you use non-ASCII characters; for 8-bits encodings only, does not work with UTF-8
  frame=single,                    % adds a frame around the code
  keepspaces=true,                 % keeps spaces in text, useful for keeping indentation of code (possibly needs columns=flexible)
  keywordstyle=\color{blue},       % keyword style
  language=Octave,                 % the language of the code
  morekeywords={*,...},            % if you want to add more keywords to the set
  numbers=left,                    % where to put the line-numbers; possible values are (none, left, right)
  numbersep=5pt,                   % how far the line-numbers are from the code
  numberstyle=\tiny\color{mygray}, % the style that is used for the line-numbers
  rulecolor=\color{black},         % if not set, the frame-color may be changed on line-breaks within not-black text (e.g. comments (green here))
  showspaces=false,                % show spaces everywhere adding particular underscores; it overrides 'showstringspaces'
  showstringspaces=false,          % underline spaces within strings only
  showtabs=false,                  % show tabs within strings adding particular underscores
  stepnumber=2,                    % the step between two line-numbers. If it's 1, each line will be numbered
  stringstyle=\color{mymauve},     % string literal style
  tabsize=2,                       % sets default tabsize to 2 spaces
  title=\lstname                   % show the filename of files included with \lstinputlisting; also try caption instead of title
}

%opening
\title{Übungsblatt 2}
\author{Uli Köhler (10580373), Tobias Harrer (10575835)}
\begin{document}

\maketitle
\section*{Aufgabe 1}%U

Der gegebenen Term hat offensichtlich nach der gegebenen Definition $5n$ Operationen, d.h. die Größe $5n$. Zu zeigen ist, dass bei der wiederholten Anwendung nur der DeMorganschen Regeln aus Blatt 1, Aufgabe 2 (d.h. der Konstruktion aus selbiger Aufgabe).

Dazu sei nur der Term ($(x_i \wedge \overline{y_i}) \vee (y_i \wedge \overline{x_i})$) betrachtet, der im gegebenen Term exakt $n$ mal vorkommt.
Eine Anwendung der DeMorganschen Regeln ergibt $\neg(x_i \wedge \overline{y_i}) \wedge \neg(\overline{x_i} \wedge y_i)$. Es ist trivialerweise sichtbar, dass sich die Größe der Formel somit um zwei Einheiten erhöht hat.

Eine weitere Anwendung auf die beiden Teilterme ergibt:\\
$\neg(\neg x_i \vee \neg\neg y_i) \wedge \neg(\neg \neg x_i \vee \neg x_i)$. Somit hat sich die Größe der Formel wiederum um zwei Einheiten erhöht, also insgesamt um 4 Einheiten.

Für jeden weiteren Term aus $\vee_{i=1}^{n}$ führt nun durch wiederholte Anwendung von DeMorgan dazu, dass eben die Anzahl der Symbole, also die Größe der Formel, insgesamt exponentiell wächst, da für jeden hinzugefügten Term die deMorganschen Regeln öfter angewendet werden müssen und somit nicht mehr eine polynomielle Ausgabegröße (der Reduktion, d.h. Eingabegröße von CNF-SAT) erzeugen können.

Daraus kann geschlussfolgert werden, dass für den Reduktionsbeweis eben die Regeln aus Blatt 1, Aufgabe 2 nicht mehr ausreichen. $SAT \leq_p CNF-SAT$ kann also nur mit dieser Methode zur Reduktion weder bewiesen noch widerlegt werden.

\section*{Aufgabe 2}%U
Zuerst muss gezeigt werden: $Erfüllbarkeit(F_1\vee F2) \leftrightarrow Erfüllbarkeit((F_1\vee x) \wedge (F_2 \vee \overline{x}))$:\\
Für $x = False$: $(F_1\vee x) \wedge (F_2 \vee \overline{x}) = (F_1\vee False) \wedge (F_2 \vee True) = F1 \wedge True = F1$\\
\textrightarrow\ ist erfüllbar genau dann wenn $F1$ erfüllbar ist

Für $x = True$: $(F_1\vee x) \wedge (F_2 \vee \overline{x}) = (F_1\vee True) \wedge (F_2 \vee False) = True \wedge F2 = F2$\\
\textrightarrow\ ist erfüllbar genau dann wenn $F2$ erfüllbar ist

x ist eine Boolesche Variable, daher ergibt sich (trivial), dass für x keine Werte außer 0 oder 1 auftreten können \textrightarrow\ 
$Erfüllbarkeit(F_1\vee F2) \leftrightarrow Erfüllbarkeit((F_1\vee x) \wedge (F_2 \vee \overline{x}))$ q.e.d.\\[5mm]

Mit diesem eben gezeigten Lemma kann nun eine Allgemeine Formel des SAT-Problem (d.h. eine Formel die nicht notwendigerweise in KNF gegeben ist) in KNF (= CNF) transformiert werden. Zu zeigen ist nun, dass diese Reduktion in Polynomialzeit funktioniert.

Es ist bekannt, dass die Eingabegröße, d.h. die Anzahl der Terme beschränkt ist.
Daraus folgt eben die Polynomialität der Reduktion, genau dann wenn das obige Lemma nicht mehrfach auf einen einzelnen Ausdruck angewendet werden kann (was zu Loops führen könnte). Dies ist offensichtlich, da sich zwar $F1 \vee F2$ auf die beiden einzelnen Terme $(F_1\vee x)$ und $(F_2 \vee \overline{x})$ anwenden ließe, nicht jedoch auf den gesamten resultierenden Term (da dort ein Konstrukt in der Form $F_x \wedge F_y$ auftritt \textrightarrow\ Anzahl der Anwendungen des Lemmas verhält sich polynomiell zur Eingabe.

Zudem ist offensichtlich, dass pro Anwendung des Lemmas lediglich eine einzige Variable ($x$) eingeführt wird, d.h. die Ausgabegröße der Reduktion (= die Eingabegröße von CNF-SAT) ist polynomiell zur ursprünglichen Eingabegröße.

\section*{Aufgabe 3}%T
$n=20000$, $\delta = 0.05$, $X :=$ "Münze zeigt Kopf",
\subsection*{a)}
%a) Wahrscheinlichkeit für Kopf 0.5
$p = 0.5$, $E(X) = n*p = 10000$\newline
X kommt mind. $5\%$ häufiger als der EW vor:\newline
$Ws(X \geq (1+\delta) * E(X)) \leq (\frac{e^{\delta}}{(1+\delta)^{1+\delta}})^{E(X)}$\newline
$=(\frac{e^{-0.05}}{(1.05)^{1.05}})^{10000} = 4.567*10^{-6}$\newline
$\leq exp(-\frac{E(X)*\delta^2}{3}) = exp(-\frac{10000*0,05^2}{3}) = 2.4*10^{-4}$\newline \newline
X kommt mind. $5\%$ weniger als der EW vor:\newline
$Ws(X \leq (1-\delta) * E(X)) \leq (\frac{e^{-\delta}}{(1-\delta)^{1+-\delta}})^{E(X)}$\newline
$=(\frac{e^{-0.05}}{(0.95)^{0.95}})^{10000} = 0.00000301$\newline
$\leq exp(-\frac{E(X)*\delta^2}{2}) = exp(-\frac{10000*0,05^2}{2}) = 0,000003727$\newline
%Zweite Formel
\subsection*{b)}
%b) Wahrscheinlichkeit für Kopf 0.1
$p = 0.1$, $E(X) = n*p = 2000$\newline
X kommt mind. $5\%$ häufiger als der EW vor:\newline
$Ws(X \geq (1+\delta) * E(X)) \leq (\frac{e^{\delta}}{(1+\delta)^{1+\delta}})^{E(X)}$\newline
$=(\frac{e^{0.05}}{(1.05)^{1.05}})^{2000} = 0.08549095 = 8.55\%$\newline
$\leq exp(-\frac{E(X)*\delta^2}{3}) = exp(-\frac{2000*0.05^2}{3}) = 0.188875603 = 18.89\%$\newline \newline
X kommt mind. $5\%$ weniger als der EW vor:\newline
$Ws(X \leq (1-\delta) * E(X)) \leq (\frac{e^{-\delta}}{(1-\delta)^{1+-\delta}})^{E(X)}$\newline
$=(\frac{e^{-0.05}}{(0.95)^{0.95}})^{2000} = 0.078650549 = 7.84\%$\newline
$\leq exp(-\frac{E(X)*\delta^2}{2}) = exp(-\frac{2000*0,05^2}{2}) = 0.082084999 = 8.2\%$\newline

\section*{Aufgabe 4}%T

\end{document}
