\documentclass[a4paper,10pt,oneside,reqno]{scrartcl}
\usepackage[utf8]{inputenc}
\usepackage{listingsutf8}
\usepackage[pdftex]{graphicx}
%\usepackage[ngerman]{babel}
\usepackage{url}
\usepackage{hyperref}
\usepackage{amsfonts}
\usepackage{amssymb}
\usepackage{amsmath}
\usepackage{tikz}
\usepackage{listings}
\usepackage{pdfpages}
\usepackage{textcomp}
\usepackage{amsmath}
%\usepackage{lipsum}%%%%%%%%%%%%%%%%%%%%%%%%%%%%%%%%%%%%%%%%%%%%%%%%%%%%%%%%%%%%%%%%%%%%%%%%%%%%%%%%%%%%
\usepackage{mathtools}
\usepackage{amsfonts}
\usepackage{amssymb}
\usepackage{amsmath}
\usepackage{booktabs}
%usepackage[utf8x]{inputenc}
\usepackage[T1]{fontenc}
\usepackage{lmodern} %Latin modern = enhanced CM font
\usepackage{xspace} %Space enhancements
%\usepackage{algorithmic}%%%%%%%%%%%%%%%%%%%%%%%%%%%%%%%%%%%%%%%%%%%%%%%%%%%%%%%%%%%%%%%%%%%%%%%%%%%%%%%%%%
%\usepackage{algpseudocode}

%opening
\title{Übungsblatt 5}
\author{Uli Köhler (10580373), Tobias Harrer (10575835)}
\begin{document}

\maketitle

\section*{Aufgabe 1}
Sei $S=\{s_w, s_{v_i} : i \in [1:m]\}$. $s_v$ ist optimaler Steinerstring $\Leftrightarrow^{Def. 6.47}$ $E_S(s_v)=min\{E_S(s') : s' \in S\}$,
d.h. die kummulierte Distanz von $s_v$ zu allen anderen $s' \in S$ ist minimal gegenüber allen Distanzen $E(s')$. Dass $E(s_v)$
bezüglich zu allen $s_{v_i}$ der String mit min. Konsensusfehler ist, folgt daraus, dass der zu $s_v$ gehörige Knoten zwischen
seinem Vater und seinen Kindern steht. Es kann weder die kummulierte Distanz eines Kindknoten , noch die des Vaterknoten zu allen
anderen kleiner sein als $E(s_v)$. Falls $v$ keinen Vater hat und somit die Wurzel von $T$ ist, ist es offensichtlich, dass $s_v$
die beste Konsensus-Sequenz für alle $s_{v_i}$ ist. Falls $v$ nicht die Wurzel ist, kann die zugehörige Sequenz des Vaters $w$
oder die eines Kindknotens $v_i$ trotzdem nicht die Konsensus-Sequenz sein, da $v$ im Graphen zwischen diesen steht.

\section*{Aufgabe 2}
Siehe schriftlich

\section*{Aufgabe 3}
Siehe schriftlich

\section*{Aufgabe 4}
Der Algorithmus hat vermutlich eine Güte $\leq 3$, da wir auf dem letzten Übungsblat gezeigt haben, dass die "mittlere"
Distanz (M(k+1)/2) höchstens so groß ist wie das Dreifache der kleinesten Distanz (M(1)) ist.

\end{document}
