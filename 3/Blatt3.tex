\documentclass[a4paper,10pt,oneside,reqno]{scrartcl}
\usepackage[utf8]{inputenc}
\usepackage{listingsutf8}
\usepackage[pdftex]{graphicx}
%\usepackage[ngerman]{babel}
\usepackage{url}
\usepackage{hyperref}
\usepackage{amsfonts}
\usepackage{amssymb}
\usepackage{amsmath}
\usepackage{tikz}
\usepackage{listings}
\usepackage{textcomp}
\usepackage{amsmath}
%\usepackage{lipsum}%%%%%%%%%%%%%%%%%%%%%%%%%%%%%%%%%%%%%%%%%%%%%%%%%%%%%%%%%%%%%%%%%%%%%%%%%%%%%%%%%%%%
\usepackage{mathtools}
\usepackage{amsfonts}
\usepackage{amssymb}
\usepackage{amsmath}
\usepackage{booktabs}
%usepackage[utf8x]{inputenc}
\usepackage[T1]{fontenc}
\usepackage{lmodern} %Latin modern = enhanced CM font
\usepackage{xspace} %Space enhancements
%\usepackage{algorithmic}%%%%%%%%%%%%%%%%%%%%%%%%%%%%%%%%%%%%%%%%%%%%%%%%%%%%%%%%%%%%%%%%%%%%%%%%%%%%%%%%%%
\usepackage{algpseudocode}
\renewcommand{\thefootnote}{\fnsymbol{footnote}}
\definecolor{mygreen}{rgb}{0,0.6,0}
\definecolor{mygray}{rgb}{0.5,0.5,0.5}
\definecolor{mymauve}{rgb}{0.58,0,0.82}
\lstset{ %
  backgroundcolor=\color{white},   % choose the background color; you must add \usepackage{color} or \usepackage{xcolor}
  basicstyle=\footnotesize,        % the size of the fonts that are used for the code
  breakatwhitespace=false,         % sets if automatic breaks should only happen at whitespace
  breaklines=true,                 % sets automatic line breaking
  captionpos=b,                    % sets the caption-position to bottom
  commentstyle=\color{mygreen},    % comment style
  deletekeywords={...},            % if you want to delete keywords from the given language
  escapeinside={\%*}{*)},          % if you want to add LaTeX within your code
  extendedchars=true,              % lets you use non-ASCII characters; for 8-bits encodings only, does not work with UTF-8
  frame=single,                    % adds a frame around the code
  keepspaces=true,                 % keeps spaces in text, useful for keeping indentation of code (possibly needs columns=flexible)
  keywordstyle=\color{blue},       % keyword style
  language=Octave,                 % the language of the code
  morekeywords={*,...},            % if you want to add more keywords to the set
  numbers=left,                    % where to put the line-numbers; possible values are (none, left, right)
  numbersep=5pt,                   % how far the line-numbers are from the code
  numberstyle=\tiny\color{mygray}, % the style that is used for the line-numbers
  rulecolor=\color{black},         % if not set, the frame-color may be changed on line-breaks within not-black text (e.g. comments (green here))
  showspaces=false,                % show spaces everywhere adding particular underscores; it overrides 'showstringspaces'
  showstringspaces=false,          % underline spaces within strings only
  showtabs=false,                  % show tabs within strings adding particular underscores
  stepnumber=2,                    % the step between two line-numbers. If it's 1, each line will be numbered
  stringstyle=\color{mymauve},     % string literal style
  tabsize=2,                       % sets default tabsize to 2 spaces
  title=\lstname                   % show the filename of files included with \lstinputlisting; also try caption instead of title
}

%opening
\title{Übungsblatt 2}
\author{Uli Köhler (10580373), Tobias Harrer (10575835)}
\begin{document}

\maketitle
\section*{Aufgabe 1}%U

Zu zeigen ist: $E3SAT \in {NP} \leftrightarrow \text{Zertifikat ist in Polynomialzeit verifizierbar}$

Statt dies direkt zu zeigen wird hier gezeigt dass $KNF-Sat \leq_p E3SAT$, also dass eine beliebige KNF-Formel in Polynomialzeit in 3-konjunktive Normalform umgewandelt werden kann.

Es sein nun die Transformation $\tau(k)$ von $KNF-SAT \rightarrow E3SAT$ wie Folgt definiert, wobei k eine Klausel, d.h. ein $K_i$ aus $\wedge_{i=0}^n K_i$ ($k = (x_i) \rightarrow \text{foobar}$ bedeutet: Wende die Regel foobar an, wenn k die Form $(x_i)$ hatn):
\begin{itemize}
 \item  $k = (x_i) \rightarrow \tau(k) := (x_i \vee x_i \vee x_i)$ Korrektheit: Trivial, Laufzeit: Obviously polnomiell (Zahl der Klauseln bleibt gleich)
 \item  $k = (x_i \vee x_{i+1}) \rightarrow \tau(k) :=  (x_i \vee x_{i+1} \vee x_i)$  \textbf{Korrektheit}: Trivial, da $\vee$ kommutativ und $x_i \vee x_i$ Laufzeit: Anzahl der Klauseln bleibt gleich, also \textbf{Polynomialzeit}
 \item $k = (x_i \vee x_{i+1} \vee x_{i+2}) \rightarrow \tau(k) := (x_i \vee x_{i+1} \vee x_{i+2})$ (Klausel ist bereits in 3-KNF. Die  \textbf{Korrektheit} ist also trivial. Anzahl der Klauseln bleibt gleich, also \textbf{Polynomialzeit})
 \item Allgemeiner Fall wenn die Anzahl der Literale in dieser Klausel $(=n)$ mit $n > 3$ ist: $k = \bigvee_{l=0}^{n} x_l \rightarrow \tau(k) := ( (x_1 \vee x_2 \vee y_1) \wedge (\bigwedge_{l=n-2}^{l=1} (\overline{y_l} \vee x_{l + 2} \vee y_{l+1})) \wedge (\overline{y_{n-3}} \vee x_{n-1} \vee x_{n})$ 
Hierbei sind die $y_i$ jeweils neue Variablen, die im Rest der Formel nicht vorkommen (für jede Anwendung des allgemeinen Falls müssen getrennte $y_i$ erzeugt werden). \textbf{Polynomialzeit}: Trivial, da $< 2n$ Klauseln erzeugt werden. Korrektheit siehe unten.
\end{itemize}

Für die Gesamte Gleichung muss noch gezeigt werden, dass die Erfüllbarkeit der 3CNF äquivalent zur originalen CNF ist.

Dafür muss gezeigt werden dass es Variablenbelegungen für die neuen Variablen $y_i$ gibt, sodass die Erfüllbarkeit äquivalent zur originalen Form ist. Für die ersten beiden Fälle wurde dies bereits gezeigt, außerdem gibt es hier keine $y_i$, es muss also für den dritten Fall gezeigt wurden.

Für Belegungen der Formel F mit mindestens einem wahren literal, sei $k$ der Index (1-based) des ersten wahren literals.

Ist $j=1$ oder $j=2$, dann können aufgrund von $(x_1 \vee x_2 \vee y_1)$ alle $y$ auf $False$ festgesetzt werden.
Äquivalentes gilt, wenn $i=n$ oder $i=n-1$ aufgrund des letzten Terms $(\overline{y_{n-3}} \vee x_{n-1} \vee x_{n})$, allerdings müssen hier alle $y$ auf $True$ festgesetzt werden, damit auch die anderen Klauseln $True$ bleiben.

Ansonsten, d.h. im Allgemeinen Fall, d.h. mit $2 < j < n-1$, müssen alle $y_i$ mit $i < (j-i)$ gleich $True$ gesetzt werden. Alle übrigen $y_i$ werden auf $False$ gesetzt. Dadurch sind die Terme vor $x_j$ durch die y-Terme erfüllbar.

Die Laufzeit des gesamten Algorithmus ist wie oben bewiesen, Polynomiell, genauergesagt $O(n^2)$, da maximal $n-3$ neue Variablen pro Klausel eingefügt wurden.

Dadurch ist die KNF in 3KNF polynomiell transformierbar $\rightarrow KNF-Sat \leq_p E3SAT \rightarrow E3SAT \in NP$

\section*{Aufgabe 2}%U
Sei $P = (I, S, \mu, min)$ ein $\mathrm{NP}$-Hartes, polynomiell beschränktes Minimierungsproblem. Wir nehmen ferner o.b.d.a. an, es gäbe ein echt polynomielles Approximationsschema A für P. Dann gibt es ein Polynom $p$, so dass die Laufzeit von $A(\epsilon)$ durch $p(\lVert x \rVert + \epsilon^{-1})$ beschränkt ist, wobei $\epsilon$ sich auf die Approximationsgüte bezieht. Da $P$ polynomiell beschränkt ist, gibt es ein weiteres Polynom $q$, so dass für jede Eingabe $x\in I$ gilt $\mu^*(x) \geq q(\lVert x \rVert)$. Wir wählen ein $\epsilon := 1/(1+q(\lVert x\rVert))$, dann liefert $A(\epsilon)$ für jedes $x$ eine $(1 + 1/q(\lVert x\rVert))$-Approximation der optimalen
Lösung, d.h. es gilt:
$\frac{\mu(x,A(\epsilon,x))}{\mu^*(x)} \leq 1 + \epsilon = \frac{q(\lVert x\rVert) + 1}{q(\lVert x\rVert)}$
Daraus ergibt sich \begin{align}
              \mu(x,A(\epsilon,x)) &\leq \mu^*(x) + \epsilon\mu^*(x) \\1
              & = \mu^*(x) + \frac{\mu^*(x) + 1 - 1}{q(\lVert x\rVert) + 1} \\
              & \text{da } \mu^*(x) \leq q(\lVert x\rVert) \\
              & \leq \mu^*(x) + \frac{q(\lVert x\rVert)}{q(\lVert x\rVert) + 1} \\
              & < \mu^*(x) + 1
             \end{align}

Da $\mu^*(x)$ per Def. ganzzahlig ist folgt daraus dass $A(\epsilon)$ eine optimale Lösung liefert. Daraus folgt direkt der zu beweisende Satz
\section*{Aufgabe 3}%T

\section*{Aufgabe 4}%T

\end{document}
