\documentclass[a4paper,10pt,oneside,reqno]{scrartcl}
\usepackage[utf8]{inputenc}
\usepackage{listingsutf8}
\usepackage[pdftex]{graphicx}
%\usepackage[ngerman]{babel}
\usepackage{url}
\usepackage{hyperref}
\usepackage{amsfonts}
\usepackage{amssymb}
\usepackage{amsmath}
\usepackage{tikz}
\usepackage{listings}
\usepackage{textcomp}
\usepackage{amsmath}
%\usepackage{lipsum}%%%%%%%%%%%%%%%%%%%%%%%%%%%%%%%%%%%%%%%%%%%%%%%%%%%%%%%%%%%%%%%%%%%%%%%%%%%%%%%%%%%%
\usepackage{mathtools}
\usepackage{amsfonts}
\usepackage{amssymb}
\usepackage{amsmath}
\usepackage{booktabs}
%usepackage[utf8x]{inputenc}
\usepackage[T1]{fontenc}
\usepackage{lmodern} %Latin modern = enhanced CM font
\usepackage{xspace} %Space enhancements
%\usepackage{algorithmic}%%%%%%%%%%%%%%%%%%%%%%%%%%%%%%%%%%%%%%%%%%%%%%%%%%%%%%%%%%%%%%%%%%%%%%%%%%%%%%%%%%
\usepackage{algpseudocode}
\renewcommand{\thefootnote}{\fnsymbol{footnote}}
\definecolor{mygreen}{rgb}{0,0.6,0}
\definecolor{mygray}{rgb}{0.5,0.5,0.5}
\definecolor{mymauve}{rgb}{0.58,0,0.82}
\lstset{ %
  backgroundcolor=\color{white},   % choose the background color; you must add \usepackage{color} or \usepackage{xcolor}
  basicstyle=\footnotesize,        % the size of the fonts that are used for the code
  breakatwhitespace=false,         % sets if automatic breaks should only happen at whitespace
  breaklines=true,                 % sets automatic line breaking
  captionpos=b,                    % sets the caption-position to bottom
  commentstyle=\color{mygreen},    % comment style
  deletekeywords={...},            % if you want to delete keywords from the given language
  escapeinside={\%*}{*)},          % if you want to add LaTeX within your code
  extendedchars=true,              % lets you use non-ASCII characters; for 8-bits encodings only, does not work with UTF-8
  frame=single,                    % adds a frame around the code
  keepspaces=true,                 % keeps spaces in text, useful for keeping indentation of code (possibly needs columns=flexible)
  keywordstyle=\color{blue},       % keyword style
  language=Octave,                 % the language of the code
  morekeywords={*,...},            % if you want to add more keywords to the set
  numbers=left,                    % where to put the line-numbers; possible values are (none, left, right)
  numbersep=5pt,                   % how far the line-numbers are from the code
  numberstyle=\tiny\color{mygray}, % the style that is used for the line-numbers
  rulecolor=\color{black},         % if not set, the frame-color may be changed on line-breaks within not-black text (e.g. comments (green here))
  showspaces=false,                % show spaces everywhere adding particular underscores; it overrides 'showstringspaces'
  showstringspaces=false,          % underline spaces within strings only
  showtabs=false,                  % show tabs within strings adding particular underscores
  stepnumber=2,                    % the step between two line-numbers. If it's 1, each line will be numbered
  stringstyle=\color{mymauve},     % string literal style
  tabsize=2,                       % sets default tabsize to 2 spaces
  title=\lstname                   % show the filename of files included with \lstinputlisting; also try caption instead of title
}

%opening
\title{Übungsblatt 4}
\author{Uli Köhler (10580373), Tobias Harrer (10575835)}
\begin{document}

\maketitle
\section*{Aufgabe 1}%U

\section*{Aufgabe 2}%U

\section*{Aufgabe 3}%T
Als erstes muss das Zentrum des Baums/Sterns gefunden werden. Nach Def. 6.36 ist der Center-String $s_c$ 
so zu wählen, dass $\Sigma_{j=1}^k d(s_c, s_j)$ minimal ist. Für die gegeben Sequenzen ergeben sich
zusammen mit den zugehörigen Distanzen folgende Summen:
\begin{enumerate}
 \item $\Sigma_{j=1}^4 d(s_j, s_1) = 0 + 3 + 3 + 3 = 9$
 \item $\Sigma_{j=1}^4 d(s_j, s_2) = 3 + 0 + 2 + 4 = 9$
 \item $\Sigma_{j=1}^4 d(s_j, s_3) = 3 + 2 + 0 + 3 = 8$
 \item $\Sigma_{j=1}^4 d(s_j, s_4) = 3 + 4 + 3 + 0 = 10$
\end{enumerate}
Daraus folgt, dass $s_3$ die geringste Distanz zu allen anderen Sequenzen hat. Mit meiner GoBi-
Implementierung (/home/proj/biocluster/praktikum/genprakt-ws13/abgaben/assignment1/harrer/gotoh.jar) bekomme ich die folgenden Alignments:\newline
\texttt{
s3: CAATG\newline
s1: CGA-A\newline\newline
s3: CAATG-\newline
s2: CAGTGA\newline\newline
s3: C-AATG\newline
s4: CGGATT\newline}

Gemäß dem Hinweis aus der letzen Übungsbesprechung wird das MSA schrittweise vom Zentrum aus aufgebaut.
\texttt{s3: CAATG\newline
s1: CGA-A\newline\newline
s3: CAATG-\newline
s1: CGA-A-\newline
s2: CAGTGA
\newline\newline
s3: C-AATG-\newline
s1: C-GA-A-\newline
s2: C-AGTGA\newline
s4: CGGATT-}

\section*{Aufgabe 4}%T
Annahme: $M(\lfloor \frac{k+1}{2}\rfloor) > 3M$, also 
$M(\lfloor \frac{k+1}{2}\rfloor) = \Sigma_{j=1}^k d(s_{\lfloor \frac{k+1}{2}\rfloor}, s_j)$ größer als
das dreifache von $M$ sein, wobei nach Def. $M = M(1)$ ist. $M(\lfloor \frac{k+1}{2}\rfloor)$ ist das


\end{document}


