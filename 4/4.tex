\documentclass[a4paper,10pt,oneside,reqno]{scrartcl}
\usepackage[utf8]{inputenc}
\usepackage{listingsutf8}
\usepackage[pdftex]{graphicx}
%\usepackage[ngerman]{babel}
\usepackage{url}
\usepackage{hyperref}
\usepackage{amsfonts}
\usepackage{amssymb}
\usepackage{amsmath}
\usepackage{tikz}
\usepackage{listings}
\usepackage{textcomp}
\usepackage{amsmath}
%\usepackage{lipsum}%%%%%%%%%%%%%%%%%%%%%%%%%%%%%%%%%%%%%%%%%%%%%%%%%%%%%%%%%%%%%%%%%%%%%%%%%%%%%%%%%%%%
\usepackage{mathtools}
\usepackage{amsfonts}
\usepackage{amssymb}
\usepackage{amsmath}
\usepackage{booktabs}
%usepackage[utf8x]{inputenc}
\usepackage[T1]{fontenc}
\usepackage{lmodern} %Latin modern = enhanced CM font
\usepackage{xspace} %Space enhancements
%\usepackage{algorithmic}%%%%%%%%%%%%%%%%%%%%%%%%%%%%%%%%%%%%%%%%%%%%%%%%%%%%%%%%%%%%%%%%%%%%%%%%%%%%%%%%%%
\usepackage{algpseudocode}
\renewcommand{\thefootnote}{\fnsymbol{footnote}}
\definecolor{mygreen}{rgb}{0,0.6,0}
\definecolor{mygray}{rgb}{0.5,0.5,0.5}
\definecolor{mymauve}{rgb}{0.58,0,0.82}
\lstset{ %
  backgroundcolor=\color{white},   % choose the background color; you must add \usepackage{color} or \usepackage{xcolor}
  basicstyle=\footnotesize,        % the size of the fonts that are used for the code
  breakatwhitespace=false,         % sets if automatic breaks should only happen at whitespace
  breaklines=true,                 % sets automatic line breaking
  captionpos=b,                    % sets the caption-position to bottom
  commentstyle=\color{mygreen},    % comment style
  deletekeywords={...},            % if you want to delete keywords from the given language
  escapeinside={\%*}{*)},          % if you want to add LaTeX within your code
  extendedchars=true,              % lets you use non-ASCII characters; for 8-bits encodings only, does not work with UTF-8
  frame=single,                    % adds a frame around the code
  keepspaces=true,                 % keeps spaces in text, useful for keeping indentation of code (possibly needs columns=flexible)
  keywordstyle=\color{blue},       % keyword style
  language=Octave,                 % the language of the code
  morekeywords={*,...},            % if you want to add more keywords to the set
  numbers=left,                    % where to put the line-numbers; possible values are (none, left, right)
  numbersep=5pt,                   % how far the line-numbers are from the code
  numberstyle=\tiny\color{mygray}, % the style that is used for the line-numbers
  rulecolor=\color{black},         % if not set, the frame-color may be changed on line-breaks within not-black text (e.g. comments (green here))
  showspaces=false,                % show spaces everywhere adding particular underscores; it overrides 'showstringspaces'
  showstringspaces=false,          % underline spaces within strings only
  showtabs=false,                  % show tabs within strings adding particular underscores
  stepnumber=2,                    % the step between two line-numbers. If it's 1, each line will be numbered
  stringstyle=\color{mymauve},     % string literal style
  tabsize=2,                       % sets default tabsize to 2 spaces
  title=\lstname                   % show the filename of files included with \lstinputlisting; also try caption instead of title
}

%opening
\title{Übungsblatt 4}
\author{Uli Köhler (10580373), Tobias Harrer (10575835)}
\begin{document}

\maketitle
\section*{Aufgabe 1}%U

\section*{Aufgabe 2}%U
\paragraph{Datenstruktur:}
Die Datenstruktur sei eine verlinkte Liste von dynamisch erweiterbaren Arrays, z.B. \textit{LinkedList<LinkedList<Char> > alignments} in Java.
Die (äußere) verlinkte Liste enthält die einzelnen, durch Gaps unterbrochenen Alignments, die durch Strings unterbrochen werden. Zudem wird eine Liste an noch zu iterierenden Baumknoten angegeben, z.B. \textit{LinkedList<Node> unprocessedNodes}. Außerdem wird eine Assoziative Struktur zur Zuordnung von einem Knoten zur diesem Knoten entsprechenden Sequenz (\textit{ArrayList<Char>} die in der oben genannten Sequenzliste enthalten ist) gespeichert, z.B. in Java \textit{HashMap<Node, ArrayList<Char> > nodeToAlignmentSequence}. Wichtig ist hier, dass die Werte dieses Mappings auf dieselben Instanzen wie oben verweisen.

Das Alignment wird nun wie Folgt erstellt, sei $n$ die Anzahl der Sequenzen und $k$ die maximale Länge der Sequenzen: 

Über den gegebenen Baum wird wie folgt iteriert:
\textbf{Prozedur A} Für den aktuellen Knoten (Start: Wurzelknoten) werden alle Kindknoten aufgelistet (hat der Knoten keine Kindknoten, wird die Prozedur sofort beendet). Jeder dieser Kindknoten wird an die Liste der noch zu iterierenden Knoten angefügt. Es ergeben sich nun Paare aus dem aktuellen Knoten und einem Kindknoten. Ist die Liste an Alignments leer (d.h. ist der aktuelle Knoten der Wurzelknoten), so wird an die Liste der Alignments die Sequenz des Wurzelknotens angefügt und in \textit{nodeToAlignmentSequence} der Eintrag \textit{Wurzelknoten \textrightarrow eben in alignments eingefügte Sequenz} eingefügt. Für jedes Paar aus aktuellem Knoten und seinem Kindknoten wird nun die Prozedur B ausgeführt. Zudem werden alle Kindknoten an \textit{unprocessedNodes} angefügt. Der erste Knoten aus \textit{unprocessedNodes} wird nun entfernt und Prozedur A für diesen Knoten (als aktuellen Knoten) ausgeführt, bis \textit{unprocessedNodes} leer ist (\textrightarrow Termination).\\[5mm]

\textbf{Prozedur B}. Eingabe: Paar aus Elternknoten $k_e$ und Kindknoten $k_k$. Dein beiden Knoten seien im Baum die Sequenzen $s_a$ und $s_b$ zugeordnet. Zuerst wird für diese beiden Sequenzen eine optimales paarweises Alignment berechnet $\mathcal{O}(n^2)$. Aufgrund der vorhergehenden Schritte ist bekannt, dass sich die Sequenz des Elternknotens bereits in \textit{alignments} und auch in in \textit{nodeToAlignmentSequence} vorhanden ist -- das Alignment für $s_e$ ist  	. Für jedes Zeichen des soeben berechneten paarweisen optimalen Alignments wird nun wie folgt vorgegangen:

Wurde in die Sequenz $s_e$ beim paarweisen Alignmenteine Gap eingefügt, so wird in alle Alignments (\textrightarrow $\mathcal{O}(k)$ an der entsprechenden Position eine Gap eingefügt. Ansonsten wird das Zeichen an der entsprechenden Position im paarweisen Alignment von $s_k$ übernommen.

Danach wird für alle Positionen, an denen das Alignment für $s_e$ bereits eine Gap enthielt, auch in das Alignment der Kindsequenz eine Gap eingefügt. Das resultierende Alignment wird nun an \textit{alignments} angefügt und das Mapping \textit{nodeToAlignmentSequence} wird entsprechend upgedated.

\textbf{Einstiegspunkt} Am Anfang wird Prozedur A mit dem Wurzelknoten als aktuellem Knoten ausgeführt.

\textit{Korrektheit} Die Korrektheit wurde bereits im Skript bewiesen.

\textit{Termination} Da nur die Kindknoten in die Liste \textit{unprocessedNodes} eingefügt werden, und bei einer Iteration der Prozedur A einer der Knoten aus der Liste entfernt wird, wird Prozedur A für jeden Knoten im Baum genau einmal ausgeführt. Da Bäume \textit{per definitionem} keine Loops haben dürfen, wird Prozedur A für jeden Knoten genau einmal ausgeführt.

\textit{Laufzeit}
Die Folgenden Faktoren spielen eine Rolle für die Laufzeit (Hinweis: Insertion aus Sicht von $s_e$, d.h. Gap ist auf $s_k$):
\begin{itemize}
 \item Das Einfügen in die Listen bzw. Maps sowie das Abrufen der Alignment-Referenz für einen gegebenen Knoten wird als konstant angenommen. Einige mögliche Implementierungen, die z.B. immutable Strings als Alignments nutzen, haben allerdings eine $\mathcal{O}(k)$-Laufzeit für einige Operation. Es wird nicht davon ausgegangen, dass der Programmierer eine solche suboptimale Ausprägung implementiert.
 \item Prozedur A wird für jede Sequenz exakt einmal ausgeführt (siehe auch Terminationsbeweis), benötigt (abgesehen von der Ausführung von Prozedur B) aber konstante Laufzeit
 \item Prozedur B wird (n-1) mal ausgeführt, da es nicht für den Wurzelknoten, aber aufgrund der Baumstruktur für alle anderen Knoten genau ausgeführt wird
 \item Die Berechnung eines optimalen paarweisen Alignments benötigt $\mathcal{O}(k^2)$ Zeit, wobei insgesamt (selbe Argumentation wie bei Proz. B) (n-1) Alignments ausgeführt werden.
 \item Für jede Deletion/Substitution im paarweisen Alignment wird konstante Zeit benötigt
 \item Für jede Insertion im paarweisen Alignment wird $\mathcal{O}(n)$ Zeit benötigt
 \item Da das paarweise Alignment max. $2k$ lang ist, d.h. $\mathcal{O}(k)$ wird maximal $\mathcal{O}(k*n)$ Zeit (wenn das Alignment nur aus Insertionen besteht
 \item Eine Ausführung von Prozedur B benötigt also im Worst Case $\mathcal{O}(k^3*n)$ Zeit.
 \item Da Proz. A konstante Zeit benötigt, und Proz. B wie oben gezeigt (n-1) mal ausgeführt wird, benötigt die gesamte Prozedur max. $\mathcal{O}(k^3*n^2)$ Zeit.
\end{itemize}

\section*{Aufgabe 3}%T
Als erstes muss das Zentrum des Baums/Sterns gefunden werden. Nach Def. 6.36 ist der Center-String $s_c$ 
so zu wählen, dass $\Sigma_{j=1}^k d(s_c, s_j)$ minimal ist. Für die gegeben Sequenzen ergeben sich
zusammen mit den zugehörigen Distanzen folgende Summen:
\begin{enumerate}
 \item $\Sigma_{j=1}^4 d(s_j, s_1) = 0 + 3 + 3 + 3 = 9$
 \item $\Sigma_{j=1}^4 d(s_j, s_2) = 3 + 0 + 2 + 4 = 9$
 \item $\Sigma_{j=1}^4 d(s_j, s_3) = 3 + 2 + 0 + 3 = 8$
 \item $\Sigma_{j=1}^4 d(s_j, s_4) = 3 + 4 + 3 + 0 = 10$
\end{enumerate}
Daraus folgt, dass $s_3$ die geringste Distanz zu allen anderen Sequenzen hat. Mit meiner GoBi-
Implementierung (/home/proj/biocluster/praktikum/genprakt-ws13/abgaben/assignment1/harrer/gotoh.jar) bekomme ich die folgenden Alignments:\newline
\texttt{
s3: CAATG\newline
s1: CGA-A\newline\newline
s3: CAATG-\newline
s2: CAGTGA\newline\newline
s3: C-AATG\newline
s4: CGGATT\newline}

Gemäß dem Hinweis aus der letzen Übungsbesprechung wird das MSA schrittweise vom Zentrum aus aufgebaut.
\texttt{s3: CAATG\newline
s1: CGA-A\newline\newline
s3: CAATG-\newline
s1: CGA-A-\newline
s2: CAGTGA
\newline\newline
s3: C-AATG-\newline
s1: C-GA-A-\newline
s2: C-AGTGA\newline
s4: CGGATT-}

\section*{Aufgabe 4}%T
Annahme: $M(\lfloor \frac{k+1}{2}\rfloor) > 3M$, also 
$M(\lfloor \frac{k+1}{2}\rfloor) = \Sigma_{j=1}^k d(s_{\lfloor \frac{k+1}{2}\rfloor}, s_j)$ größer als
das dreifache von $M$ sein, wobei nach Def. $M = M(1)$ ist. $M(\lfloor \frac{k+1}{2}\rfloor)$ ist das

\end{document}
