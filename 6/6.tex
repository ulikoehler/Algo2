\documentclass[a4paper,10pt,oneside,reqno]{scrartcl}
\usepackage[utf8]{inputenc}
\usepackage{listingsutf8}
\usepackage[pdftex]{graphicx}
%\usepackage[ngerman]{babel}
\usepackage{url}
\usepackage{hyperref}
\usepackage{amsfonts}
\usepackage{amssymb}
\usepackage{amsmath}
\usepackage{tikz}
\usepackage{listings}
\usepackage{pdfpages}
\usepackage{textcomp}
\usepackage{amsmath}
%\usepackage{lipsum}%%%%%%%%%%%%%%%%%%%%%%%%%%%%%%%%%%%%%%%%%%%%%%%%%%%%%%%%%%%%%%%%%%%%%%%%%%%%%%%%%%%%
\usepackage{mathtools}
\usepackage{amsfonts}
\usepackage{amssymb}
\usepackage{amsmath}
\usepackage{booktabs}
%usepackage[utf8x]{inputenc}
\usepackage[T1]{fontenc}
\usepackage{lmodern} %Latin modern = enhanced CM font
\usepackage{xspace} %Space enhancements
%\usepackage{algorithmic}%%%%%%%%%%%%%%%%%%%%%%%%%%%%%%%%%%%%%%%%%%%%%%%%%%%%%%%%%%%%%%%%%%%%%%%%%%%%%%%%%%
%\usepackage{algpseudocode}

\usepackage[paper=a4paper,includefoot,includehead,left=30mm,right=20mm,top=20mm,bottom=20mm]{geometry}
%opening
\title{Übungsblatt 5}
\author{Uli Köhler (10580373), Tobias Harrer (10575835)}
\begin{document}

\maketitle

\section*{Aufgabe 1}%U
s1, s2: \\
\begin{tabular}{c|ccccccc}
& \textbf{ } & \textbf{A} & \textbf{C} & \textbf{G} & \textbf{T} & \textbf{G} & \textbf{C}\\\hline
\textbf{ } & 0.0 & 2.0 & 4.0 & 6.0 & 8.0 & 10.0 & 12.0\\
\textbf{A} & 2.0 & 0.0 & 2.0 & 4.0 & 6.0 & 8.0 & 10.0\\
\textbf{C} & 4.0 & 2.0 & 0.0 & 2.0 & 4.0 & 6.0 & 8.0\\
\textbf{C} & 6.0 & 4.0 & 2.0 & 3.0 & 5.0 & 7.0 & 6.0\\
\textbf{T} & 8.0 & 6.0 & 4.0 & 5.0 & 3.0 & 5.0 & 7.0\\
\textbf{G} & 10.0 & 8.0 & 6.0 & 4.0 & 5.0 & 3.0 & 5.0\\
\end{tabular}
\\[2mm]s1, s3: \\
\begin{tabular}{c|ccccccc}
& \textbf{ } & \textbf{A} & \textbf{C} & \textbf{G} & \textbf{T} & \textbf{G} & \textbf{C}\\\hline
\textbf{ } & 0.0 & 2.0 & 4.0 & 6.0 & 8.0 & 10.0 & 12.0\\
\textbf{A} & 2.0 & 0.0 & 2.0 & 4.0 & 6.0 & 8.0 & 10.0\\
\textbf{G} & 4.0 & 2.0 & 3.0 & 2.0 & 4.0 & 6.0 & 8.0\\
\textbf{G} & 6.0 & 4.0 & 5.0 & 3.0 & 5.0 & 4.0 & 6.0\\
\textbf{C} & 8.0 & 6.0 & 4.0 & 5.0 & 6.0 & 6.0 & 4.0\\
\textbf{T} & 10.0 & 8.0 & 6.0 & 7.0 & 5.0 & 7.0 & 6.0\\
\textbf{T} & 12.0 & 10.0 & 8.0 & 9.0 & 7.0 & 8.0 & 8.0\\
\end{tabular}
\\[2mm]s1, s4: \\
\begin{tabular}{c|ccccccc}
& \textbf{ } & \textbf{A} & \textbf{C} & \textbf{G} & \textbf{T} & \textbf{G} & \textbf{C}\\\hline
\textbf{ } & 0.0 & 2.0 & 4.0 & 6.0 & 8.0 & 10.0 & 12.0\\
\textbf{A} & 2.0 & 0.0 & 2.0 & 4.0 & 6.0 & 8.0 & 10.0\\
\textbf{G} & 4.0 & 2.0 & 3.0 & 2.0 & 4.0 & 6.0 & 8.0\\
\textbf{C} & 6.0 & 4.0 & 2.0 & 4.0 & 5.0 & 7.0 & 6.0\\
\textbf{C} & 8.0 & 6.0 & 4.0 & 5.0 & 7.0 & 8.0 & 7.0\\
\end{tabular}
\\[2mm]s2, s3: \\
\begin{tabular}{c|cccccc}
& \textbf{ } & \textbf{A} & \textbf{C} & \textbf{C} & \textbf{T} & \textbf{G}\\\hline
\textbf{ } & 0.0 & 2.0 & 4.0 & 6.0 & 8.0 & 10.0\\
\textbf{A} & 2.0 & 0.0 & 2.0 & 4.0 & 6.0 & 8.0\\
\textbf{G} & 4.0 & 2.0 & 3.0 & 5.0 & 7.0 & 6.0\\
\textbf{G} & 6.0 & 4.0 & 5.0 & 6.0 & 8.0 & 7.0\\
\textbf{C} & 8.0 & 6.0 & 4.0 & 5.0 & 7.0 & 9.0\\
\textbf{T} & 10.0 & 8.0 & 6.0 & 7.0 & 5.0 & 7.0\\
\textbf{T} & 12.0 & 10.0 & 8.0 & 9.0 & 7.0 & 8.0\\
\end{tabular}
\\[2mm]s2, s4: \\
\begin{tabular}{c|cccccc}
& \textbf{ } & \textbf{A} & \textbf{C} & \textbf{C} & \textbf{T} & \textbf{G}\\\hline
\textbf{ } & 0.0 & 2.0 & 4.0 & 6.0 & 8.0 & 10.0\\
\textbf{A} & 2.0 & 0.0 & 2.0 & 4.0 & 6.0 & 8.0\\
\textbf{G} & 4.0 & 2.0 & 3.0 & 5.0 & 7.0 & 6.0\\
\textbf{C} & 6.0 & 4.0 & 2.0 & 3.0 & 5.0 & 7.0\\
\textbf{C} & 8.0 & 6.0 & 4.0 & 2.0 & 4.0 & 6.0\\
\end{tabular}
\\[2mm]s3, s4: \\
\begin{tabular}{c|ccccccc}
& \textbf{ } & \textbf{A} & \textbf{G} & \textbf{G} & \textbf{C} & \textbf{T} & \textbf{T}\\\hline
\textbf{ } & 0.0 & 2.0 & 4.0 & 6.0 & 8.0 & 10.0 & 12.0\\
\textbf{A} & 2.0 & 0.0 & 2.0 & 4.0 & 6.0 & 8.0 & 10.0\\
\textbf{G} & 4.0 & 2.0 & 0.0 & 2.0 & 4.0 & 6.0 & 8.0\\
\textbf{C} & 6.0 & 4.0 & 2.0 & 3.0 & 2.0 & 4.0 & 6.0\\
\textbf{C} & 8.0 & 6.0 & 4.0 & 5.0 & 3.0 & 5.0 & 7.0\\
\end{tabular}
\\[2mm]Sequence distance matrix: 
\begin{tabular}{c|cccc}
& \textbf{s1} & \textbf{s2} & \textbf{s3} & \textbf{s4}\\\hline
\textbf{s1} & 0.0 & 5.0 & 8.0 & 7.0\\
\textbf{s2} & 5.0 & 0.0 & 8.0 & 6.0\\
\textbf{s3} & 8.0 & 8.0 & 0.0 & 7.0\\
\textbf{s4} & 7.0 & 6.0 & 7.0 & 0.0\\
\end{tabular} \\

Center string: $s_2$ (Berechnung der Zeilensumme $\rightarrow$ zweite Zeile hat kleinste Zeilensumme)

Die vorherigen Berechnungen wurden geskriptet, siehe insbesondere:\\
\url{http://techoverflow.net/blog/2013/12/08/center-star-approximation-identifying-the-center-string-in-python/} und\\
\url{http://techoverflow.net/blog/2013/12/08/converting-pandas-dataframe-to-latex-tabular/}

Wie in Satz 6.59/3 angegeben konstruiert die Center-Star-Methode (Zentrum: $s_2$) eben ein Alignment das ein 2-Approximation besitzt, also kann Center-Star hier verwendet werden.

Traceback von allen Alignments mit dem Centerstring $s_2$:

s1, s2: \\
\begin{tabular}{c|ccccccc}
& \textbf{ } & \textbf{A} & \textbf{C} & \textbf{G} & \textbf{T} & \textbf{G} & \textbf{C}\\\hline
\textbf{ } & \textbf{0.0} & 2.0 & 4.0 & 6.0 & 8.0 & 10.0 & 12.0\\
\textbf{A} & 2.0 & \textbf{0.0} & 2.0 & 4.0 & 6.0 & 8.0 & 10.0\\
\textbf{C} & 4.0 & 2.0 & \textbf{0.0} & 2.0 & 4.0 & 6.0 & 8.0\\
\textbf{C} & 6.0 & 4.0 & 2.0 & \textbf{3.0} & 5.0 & 7.0 & 6.0\\
\textbf{T} & 8.0 & 6.0 & 4.0 & 5.0 & \textbf{3.0} & 5.0 & 7.0\\
\textbf{G} & 10.0 & 8.0 & 6.0 & 4.0 & 5.0 & \textbf{3.0} & \textbf{5.0}\\
\end{tabular}\\
Also:
\begin{verbatim}
ACCTG-
ACGTGC
\end{verbatim}

s2, s3: \\
\begin{tabular}{c|cccccc}
& \textbf{ } & \textbf{A} & \textbf{C} & \textbf{C} & \textbf{T} & \textbf{G}\\\hline
\textbf{ } & \textbf{0.0} & 2.0 & 4.0 & 6.0 & 8.0 & 10.0\\
\textbf{A} & 2.0 & \textbf{0.0} & 2.0 & 4.0 & 6.0 & 8.0\\
\textbf{G} & 4.0 & 2.0 & \textbf{3.0} & 5.0 & 7.0 & 6.0\\
\textbf{G} & 6.0 & 4.0 & \textbf{5.0} & 6.0 & 8.0 & 7.0\\
\textbf{C} & 8.0 & 6.0 & 4.0 & \textbf{5.0} & 7.0 & 9.0\\
\textbf{T} & 10.0 & 8.0 & 6.0 & 7.0 & \textbf{5.0} & 7.0\\
\textbf{T} & 12.0 & 10.0 & 8.0 & 9.0 & 7.0 & \textbf{8.0}\\
\end{tabular}

also:
\begin{verbatim}
AC-CTG
AGGCTT
\end{verbatim}

und

s2, s4: \\
\begin{tabular}{c|cccccc}
& \textbf{ } & \textbf{A} & \textbf{C} & \textbf{C} & \textbf{T} & \textbf{G}\\\hline
\textbf{ } & \textbf{0.0} & 2.0 & 4.0 & 6.0 & 8.0 & 10.0\\
\textbf{A} & 2.0 & \textbf{0.0} & 2.0 & 4.0 & 6.0 & 8.0\\
\textbf{G} & 4.0 & \textbf{2.0} & 3.0 & 5.0 & 7.0 & 6.0\\
\textbf{C} & 6.0 & 4.0 & \textbf{2.0} & 3.0 & 5.0 & 7.0\\
\textbf{C} & 8.0 & 6.0 & 4.0 & \textbf{2.0} & \textbf{4.0} & \textbf{6.0}\\
\end{tabular}

also:
\begin{verbatim}
A-CCTG
AGCC--
\end{verbatim}

daraus folgt nach Center Star:

\begin{verbatim}
A-C-CTG-
A-C-GTGC
A-GGCTT-
AGC-C---
\end{verbatim}

\section*{Aufgabe 2}%U

\section*{Aufgabe 3}%T


\section*{Aufgabe 4}%T

\begin{alignat}{5}
 D[s, s_1] &= (d(s_1,s_1)+0) &+{}& (d(s_1,s_2)+0) &=& 0 + 1 &= 1\\
 D[s, s_2] &= (d(s_2,s_1)+0) &+{}& (d(s_2,s_2)+0) &=& 1 + 0 &= 1\\\hline 
 D[t, s_3] &= (d(s_3,s_3)+0) &+{}& (d(s_3,s_4)+0) &=& 0 + 3 &= 3\\
 D[t, s_4] &= (d(s_4,s_3)+0) &+{}& (d(s_4,s_4)+0) &=& 3 + 0 &= 3\\\hline
 D[r, s_1] &= (d(s_1,s_1)+D[s, s_1]) &+{}& (d(s_1,s_3)+D[t, s_3]) &=& (0+1) + (1+3) &= 5\\
 D[r, s_2] &= (d(s_2,s_2)+D[s, s_2]) &+{}& (d(s_2,s_4)+D[t, s_4]) &=& (0+1) + (2+3) &= 6\\
 D[r, s_3] &= (d(s_3,s_1)+D[s, s_1]) &+{}& (d(s_3,s_3)+D[t, s_3]) &=& (1+1) + (0+3) &= 5\\
 D[r, s_4] &= (d(s_4,s_2)+D[s, s_2]) &+{}& (d(s_4,s_4)+D[t, s_4]) &=& (2+1) + (0+3) &= 6
\end{alignat}

Also: $r = s_1 \vee r = s_3 ; s = s_1 ; t = s_3$. Für $r$ sind $s_1$ und $s_3$ gleichwertig (beide haben den min. Score 5).
\end{document}
















