\documentclass[a4paper,10pt,oneside,leqno]{scrartcl}
\usepackage[utf8]{inputenc}
\usepackage{listingsutf8}
\usepackage[pdftex]{graphicx}
%\usepackage[ngerman]{babel}
\usepackage{url}
\usepackage{hyperref}
\usepackage{amsfonts}
\usepackage{amssymb}
\usepackage{amsmath}
\usepackage{tikz}
\usepackage{listings}
\usepackage{textcomp}
\usepackage{amsmath}
%\usepackage{lipsum}%%%%%%%%%%%%%%%%%%%%%%%%%%%%%%%%%%%%%%%%%%%%%%%%%%%%%%%%%%%%%%%%%%%%%%%%%%%%%%%%%%%%
\usepackage{mathtools}
\usepackage{amsfonts}
\usepackage{amssymb}
\usepackage{amsmath}
\usepackage{booktabs}
%usepackage[utf8x]{inputenc}
\usepackage[T1]{fontenc}
\usepackage{lmodern} %Latin modern = enhanced CM font
\usepackage{xspace} %Space enhancements
%\usepackage{algorithmic}%%%%%%%%%%%%%%%%%%%%%%%%%%%%%%%%%%%%%%%%%%%%%%%%%%%%%%%%%%%%%%%%%%%%%%%%%%%%%%%%%%
\usepackage{algpseudocode}
\renewcommand{\thefootnote}{\fnsymbol{footnote}}
\definecolor{mygreen}{rgb}{0,0.6,0}
\definecolor{mygray}{rgb}{0.5,0.5,0.5}
\definecolor{mymauve}{rgb}{0.58,0,0.82}
\lstset{ %
  backgroundcolor=\color{white},   % choose the background color; you must add \usepackage{color} or \usepackage{xcolor}
  basicstyle=\footnotesize,        % the size of the fonts that are used for the code
  breakatwhitespace=false,         % sets if automatic breaks should only happen at whitespace
  breaklines=true,                 % sets automatic line breaking
  captionpos=b,                    % sets the caption-position to bottom
  commentstyle=\color{mygreen},    % comment style
  deletekeywords={...},            % if you want to delete keywords from the given language
  escapeinside={\%*}{*)},          % if you want to add LaTeX within your code
  extendedchars=true,              % lets you use non-ASCII characters; for 8-bits encodings only, does not work with UTF-8
  frame=single,                    % adds a frame around the code
  keepspaces=true,                 % keeps spaces in text, useful for keeping indentation of code (possibly needs columns=flexible)
  keywordstyle=\color{blue},       % keyword style
  language=Octave,                 % the language of the code
  morekeywords={*,...},            % if you want to add more keywords to the set
  numbers=left,                    % where to put the line-numbers; possible values are (none, left, right)
  numbersep=5pt,                   % how far the line-numbers are from the code
  numberstyle=\tiny\color{mygray}, % the style that is used for the line-numbers
  rulecolor=\color{black},         % if not set, the frame-color may be changed on line-breaks within not-black text (e.g. comments (green here))
  showspaces=false,                % show spaces everywhere adding particular underscores; it overrides 'showstringspaces'
  showstringspaces=false,          % underline spaces within strings only
  showtabs=false,                  % show tabs within strings adding particular underscores
  stepnumber=2,                    % the step between two line-numbers. If it's 1, each line will be numbered
  stringstyle=\color{mymauve},     % string literal style
  tabsize=2,                       % sets default tabsize to 2 spaces
  title=\lstname                   % show the filename of files included with \lstinputlisting; also try caption instead of title
}

%opening
\title{Übungsblatt 1}
\author{Uli Köhler (10580373), Tobias Harrer (10575835)}
\begin{document}

\maketitle
\section*{Aufgabe 1}%U

\section*{Aufgabe 2}%U

\section*{Aufgabe 3}%T
%Beweisen Sie, dass wenn P ∈ NPO ist, das zu P gehörige Entscheidungsproblem in NP ist.
%Entscheidungsproblem eines Optimierungsproblems: Es soll entsch. werden, ob d. opt.Lösung besser als vorg. Schranke ist.
Geg. Optimierungsproblem P, dann ist P' = $\{x \in I : \exists y \in S(x) : \mu(x, y) = opt(\mu(x, y), B)\}$ das zugehörige Entscheidungsproblem. \newline
Zu zeigen: \begin{itemize}
     \item $\forall$ x mit P'(x) = 1 $\exists$ Zertifikat z : M(x,z) = 1
     \item $\forall$ x mit P'(x) = 0 $\forall$ Zertifikat z : M(x,z) = 0
    \end{itemize}
    
Wenn das Entscheidungsproblem ein positives Ergebnis liefert (P'(x) = 1), dann gibt es auch ein Zertifikat z, das in polynomieller Zeit M(x,z) = 1
berechnet, da per Definition von NPO in polynomieller Zeit entschieden werden kann, ob $x \in I$ ist oder nicht, und die Größe der Lösung
polynomiell in der Eingabegröße beschränkt ist. Ist P'(x) = 0, so liefert z auch M(x,z) = 0.
\section*{Aufgabe 4}%T
%(a) Zeigen Sie, dass TSP ∈ NPO.
\subsection*{a)}
Zu zeigen: \begin{itemize}
            \item  Es kann in polynomieller Zeit entschieden werden, ob $x \in I$ ist oder nicht, d.h. $I \in P$.
	    \item Die Größe jeder Lösung ist polynomiell in der Eingabegröße beschränkt, d.h. $\exists$ ein Polynom p, so dass $\forall x \in I$
	    und $\forall y \in S(x)$ gilt: $||y|| \leq p(||x||)$.
	    \item Das Maß $\mu$ ist in polynomieller Zeit berechenbar.
           \end{itemize}

%(b) Zeigen Sie, dass das zu TSP gehörige Entscheidungsproblem NP-hart ist.

\end{document}
