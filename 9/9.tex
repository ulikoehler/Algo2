\documentclass[a4paper,10pt,oneside,leqno]{scrartcl}
\usepackage[utf8]{inputenc}
\usepackage{listingsutf8}
\usepackage[pdftex]{graphicx}
%\usepackage[ngerman]{babel}
\usepackage{url}
\usepackage{hyperref}
\usepackage{amsfonts}
\usepackage{amssymb}
\usepackage{amsmath}
\usepackage{tikz}
\usepackage{listings}
\usepackage{pdfpages}
\usepackage{textcomp}
\usepackage{amsmath}
%\usepackage{lipsum}%%%%%%%%%%%%%%%%%%%%%%%%%%%%%%%%%%%%%%%%%%%%%%%%%%%%%%%%%%%%%%%%%%%%%%%%%%%%%%%%%%%%
\usepackage{mathtools}
\usepackage{amsfonts}
\usepackage{amssymb}
\usepackage{amsmath}
\usepackage{booktabs}
%usepackage[utf8x]{inputenc}
\usepackage[T1]{fontenc}
\usepackage{lmodern} %Latin modern = enhanced CM font
\usepackage{xspace} %Space enhancements
%\usepackage{algorithmic}%%%%%%%%%%%%%%%%%%%%%%%%%%%%%%%%%%%%%%%%%%%%%%%%%%%%%%%%%%%%%%%%%%%%%%%%%%%%%%%%%%
%\usepackage{algpseudocode}

\usepackage[paper=a4paper,includefoot,includehead,left=30mm,right=20mm,top=20mm,bottom=20mm]{geometry}
%opening
\title{Übungsblatt 9}
\author{Uli Köhler (10580373), Tobias Harrer (10575835)}
\begin{document}

\maketitle

\section*{Aufgabe 1}%U


\section*{Aufgabe 2}%T
Beim siebten Wurf fiel eine ``sechs''. Die Wahrscheinlichkeit dafür ist gemäß der geometrischen Verteilung:
P(X=7) = $(1-\frac{1}{6})^{7-1}*\frac{1}{6} = (\frac{5}{6})^{6}*\frac{1}{6} = 0,055816329 \approx 5,58\%$\newline
Dabei ist $H_0:$ ``Würfel ist fair'' und $H_1:$ ``Würfel benachteiligt die sechs''. Wir lehnen $H_0$ ab, falls
$P(X=N)<\alpha = 0.05$ ist. Wie wir gesehen haben ist für $N=7$ das SIgnifikanzniveau noch nicht unterschritten.
Für $N=8$ gilt hingegen P(X=8) = $(1-\frac{1}{6})^{8-1}*\frac{1}{6} = (\frac{5}{6})^{7}*\frac{1}{6} = 
0,046513608 \approx 4,67\%$, das heißt der Signifikanzpunkt liegt bei $k=8$ und der kritische Bereich $C = \{8,\infty)$.
Da das Ergebnis (7) unserer Teststatistik nahe bei, aber dennoch nicht in $C$ liegt, müssen wir $H_0$ annehmen.
(Verwendet wurde random.py statt eines Würfels, daher war das auch zu erwarten)

\section*{Aufgabe 3}%T
\subsection*{a)}
$\alpha=0.05$, $\phi_0=\frac{1}{6}$,$\phi_1=\frac{1}{7}$\newline
$L(x;\frac{1}{6}) = \frac{1}{6}*(\frac{5}{6})^{x-1}, L(x;\frac{6}{7}) = \frac{1}{7}*(\frac{1}{7}^{x-1})$.\newline
Damit ist $\Lambda(x) = \frac{L(x:\phi_0)}{L(x;\phi_1)} = \frac{7}{6}*(\frac{35}{36})^{x-1}$, und $\alpha = 
0.05 = P(\Lambda(x)\leq \lambda | \phi_0)$. Daher liegt der kritische Punkt wieder bei 8

\subsection*{b)}
analog zu a), der kritische Punkt leigt dabei >8.

\section*{Aufgabe 4}%U


\end{document}
















